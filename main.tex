\documentclass[12pt]{scrartcl}

\usepackage{amsmath, amssymb, amsthm, mathtools}

\usepackage{tikz, pgfplots}
\pgfplotsset{compat=1.17, width=10cm}
\usepgfplotslibrary{external}
\tikzexternalize

\usepackage[OT1]{fontenc}
\usepackage{mlmodern}
\usepackage{setspace}

\usepackage[backend=biber,style=numeric,sortcites=true]{biblatex}
\addbibresource{main.bib}

\usepackage{hyperref}

\title{The Bigger The Better II}
\author{Group 8-29 \thanks{Derrick Lukimin (L, 2i204), Tan Yong Yih (2i222), Wu Hao (2i324), Darren Yap (2i425)}}
\date{2022}

\begin{document}
\onehalfspacing
\maketitle
\tableofcontents

\section{Introduction}
This project aims to find an algorithm to determine
the side length of the largest square that can be
inscribed inside a convex $n$-gon. It is a continuation from
a previous project completed in 2021, The Bigger The Better. \cite{tbtb1}

\subsection{Rationale}
Do note that the definition of inscribed is such that all vertices of the square lie on the sides of the polygon.

\subsection{Research Questions}
\begin{enumerate}
	\item What is the side length of the largest square that can be inscribed in a triangle?
	\item What is the side length of the largest square that can be inscribed in a regular $n$-gon, given $n \neq 4$?
	\item What is the side length of the largest square that can be inscribed in a convex $n$-gon?
\end{enumerate}

\subsection{Project Scope}
This project will mainly focus on polygons which are convex. This allows many restrictions to be made.

\section{Research Question 1}

\subsection{Introduction}
The first research question aims to find out the side length of the largest square that can be inscribed in a triangle, given the side lengths of the triangle.

\subsection{Key insights}
\begin{enumerate}
	\item It can be seen that no more than 2 vertices of a square can lie on a single side, as a square has at most 2 vertices lying on a single line.
	\item We notice how a triangle has 3 sides, and a square has 4 vertices. In order for all the vertices to lie on the triangle, by pigeonhole principle, at least one side has at least 2 vertices lying on it.
	\item Combining the first 2 insights, we can see that 2 sides of the triangle will have 1 vertices each lying on it, while the other side will have 2 vertices lying on it.
\end{enumerate}

\subsection{Solutions}
%insert image 1 here
\(s\ =\ \frac{c}{1+\cot \angle A+\cot \angle B}\) \\
\(=\frac{c\sin \angle A}{\sin \angle A+\cos \angle A+\cot \angle B\sin \angle A}\) \\
\(=\frac{c\sin \angle A\sin \angle B}{\sin \angle A\sin \angle B+\cos \angle A\sin \angle B+\sin \angle A\cos \angle B}\) \\
\(=\frac{c\sin \angle A\sin \angle B}{\sin \angle A\sin \angle B+\sin\left(\angle A+\angle B\right)}\) \\
\(=\frac{c\sin \angle A\sin \angle B}{\sin \angle A\sin \angle B+\sin\left(180-\angle C\right)}\) \\
\(=\frac{c\sin \angle A\sin \angle B}{\sin \angle A\sin \angle B+\sin \angle C}\) \\
\(=\frac{2Rc\sin \angle A\sin \angle B}{2R\sin \angle A\sin \angle B+2R\sin \angle C}\) \\
\(=\frac{ac\sin \angle B}{a\sin \angle B+c}\) \\
\(=\frac{2Rac\sin \angle B}{2Ra\sin \angle B+2Rc}\) \\
\(=\frac{abc}{2Rc+ab}\)



\printbibliography
\end{document}
