\documentclass[12pt]{scrartcl}
\usepackage{amsmath, amssymb, amsthm, mathtools,amsthm}
\newtheorem{theorem}{Theorem}[section]
\newtheorem{corollary}{Corollary}[theorem]
\newtheorem{lemma}[theorem]{Lemma}

\usepackage{enumitem}
\usepackage{tikz, pgfplots}
\pgfplotsset{compat=1.17, width=10cm}
\usepgfplotslibrary{external}
\tikzexternalize

\usepackage{siunitx}

\usepackage[OT1]{fontenc}
\usepackage{mlmodern}
\usepackage{setspace}

\usepackage[backend=biber,style=numeric,sortcites=true]{biblatex}
\addbibresource{main.bib}
\usepackage{hyperref}

\title{The Bigger The Better II}
\author{Group 8-29 \thanks{Derrick Lukimin (L, 2i204), Tan Yong Yih (2i222), Wu Hao (2i324), Darren Yap (2i425)}}
\date{2022}

\begin{document}
\onehalfspacing
\maketitle
\tableofcontents

\section{Introduction}
This project aims to find an algorithm to determine
the side length of the largest square that can be
inscribed inside a convex $n$-gon. It is a continuation from
a previous project completed in 2021, The Bigger The Better. \cite{tbtb1}

\subsection{Definitions}
\begin{description}[font=\bfseries, leftmargin=1cm, style=nextline]
	\item[placement] A valid location, size and rotation of the square such that
		all vertices of the square lie on the edges of the polygon.
	\item[inscribed] All vertices of the square must lie on the edges of the polygon.
	\item[fit] All vertices of the square must lie within the polygon.
	\item[RQ] Research Question
\end{description}

\subsection{Research Questions}
\begin{enumerate}
	\item What is the side length of the largest square that can be inscribed in a triangle?
	\item What is the side length of the largest square that can be inscribed in a regular $n$-gon, given $n \neq 4$?
	\item What is the side length of the largest square that can be inscribed in a convex $n$-gon?
\end{enumerate}

\subsection{Project Scope}
This project will only focus on convex polygons.

\section{Literature Review}
% stuff...

\section{Research Question 1}

RQ1 aims to find out the side length of the largest square that can be inscribed in a triangle,
given the side lengths of the triangle, $a$, $b$ and $c$.

\subsection{Key Insights}
Some key insights which greatly aided in solving this problem were found.
\begin{enumerate}
	\item It can be seen that no more than two vertices of a square can lie on a single side,
	      as a square has at most two co-linear vertices.
	\item We notice how a triangle has three sides, and a square has four vertices.
	      In order for all the vertices to lie on the triangle, using the Pigeonhole Principle,
	      there will be at least one side with two vertices lying on it.
	\item Combining the above insights, there will be one vertex of the square
	      each lying on two sides of the triangle, with the other two vertices of the square
	      lying on the latter side of the triangle.
\end{enumerate}

\subsection{Solution}
A figure has been constructed for the purposes of illustrating the following proof.
\begin{figure}[htpb]
	\centering
	\includegraphics[scale=.75]{images/rq1.jpg}
	\label{fig:rq1_img}
	\caption{The figure for RQ1.}
\end{figure}

Let $s$ be the side length of the largest square that can be inscribed in a triangle,
with side lengths $a$, $b$ and $c$, and circumradius $R$.

Side $c$ can be formed with the sum of $s$, $s \cot A$ and $s \cot \angle{B}$.
Hence, we can express $s$ with the side length $c$, as well as angles $A$ and $B$.

\begin{align*}
	c & = s+s\cot \angle{A}+s\cot \angle{B}          \\
	s & = \frac{c}{1+\cot \angle{A}+\cot \angle{B}}                                                                            
\end{align*}

Both sides of the fraction can be multiplied by $\sin A \sin B$. Following which, the
Sine Addition Formula can be applied.

\begin{align*}
	s & = \frac{c \sin\angle{A}}{\sin\angle{A}+\cos\angle{A}+\cot\angle{B}\sin\angle{A}}                                       \\
	  & = \frac{c\sin\angle{A}\sin\angle{B}}{\sin\angle{A}\sin\angle{B}+\cos\angle{A}\sin\angle{B}+\sin\angle{A}\cos\angle{B}} \\
	  & = \frac{c\sin\angle{A}\sin\angle{B}}{\sin\angle{A}\sin\angle{B}+\sin\left(\angle{A}+\angle{B}\right)}                  \\
	  & = \frac{c\sin\angle{A}\sin\angle{B}}{\sin\angle{A}\sin\angle{B}+\sin\left(180-\angle C\right)}                         \\
	  & = \frac{c\sin\angle{A}\sin\angle{B}}{\sin\angle{A}\sin\angle{B}+\sin \angle C}                                         
\end{align*}

Both sides of the fraction can be multiplied by $2Rc$ and the Law of Sines can be used to simplify.

\begin{align*}
	 s & = \frac{2Rc\sin\angle{A}\sin\angle{B}}{2R\sin\angle{A}\sin \angle{B}+2R\sin \angle C}                                  \\
	  & = \frac{ac\sin\angle{B}}{a\sin\angle{B}+c}                                                                             \\
	  & = \frac{2Rac\sin\angle{B}}{2Ra\sin\angle{B}+2Rc}                                                                       \\
	  & = \frac{abc}{2Rc+ab}
\end{align*}
% We used formulae and theorems such as the sine addition formula and the law of sines to manipulate the expressions. \\

Since each of the sides of the triangle, $a$, $b$ and $c$ can be the longest side,
the maximum of the three placements can be taken as the solution, hence
\begin{equation}
	s_{\text{max}} = \max\left(\dfrac{abc}{2Rc+ab},\dfrac{abc}{2Rb+ac},\dfrac{abc}{2Ra+bc}\right)
\end{equation}

For obtuse triangles, only one placement exists, i.e. when the square lies on the longest side.
\begin{equation}
	s = \dfrac{abc}{2Rc+ab}
\end{equation}
where $c$ is the longest side.

\section{Research Question 2}

RQ2 aims to find out the side length of the largest square that can be inscribed in a convex $n$-gon,
given $n$ and the side length of the $n$-gon, $k$.

This problem can be further split into four cases.
\begin{enumerate}
	\item when \(n \equiv 0\) (mod $4$),
	\item when \(n \equiv 2\) (mod $4$),
	\item when \(n \equiv 1\) (mod $4$),
	\item when \(n \equiv 3\) (mod $4$). 
\end{enumerate}

\subsection{Case 1}
This case deals with the scenario where the number of sides in the $n$-gon is divisible by $4$.

Firstly, for the side length of the square to be maximised, more than one vertex of the square must coincide with the perimeter of the $n$-gon.
This can be easily seen, because the square can be pushed outwards in the other direction if less than two vertices of the square touch the perimeter of the $n$-gon.

Let the square be $ABCD$, and the polygon be $V_{1}V_{2}...V_{n}$. Also, let \(n = 4m\), where m is an integer.

Due to such a polygon being symmetrical both horizontally and vertically, the assumption that the centre of the polygon coincides with the centre of the square can be made. From the earlier observation, it can be assumed that one vertex of the square, $A$, lies on the perimeter of the polygon. Also, due to the polygon being symmetrical both horizontally and vertically, the opposite vertex, $C$, will also lie on the perimeter of the polygon.

\begin{figure}[htpb]
	\centering
	\includegraphics[scale=.75]{images/rq2_1_1.jpg}
	\label{fig:rq2_1_1_img}
	\caption{The figure for finding possible placements for case 1 of RQ2.}
\end{figure}

Due to the symmetry, as long as vertex B lies on the perimeter of the polygon, the square can be inscribed. If vertex B lies within the polygon, the square can be fit. We shall find the values of $\theta$ such that the square can be inscribed or fit.

Firstly, it can be seen that $V_{1}OV_{2} = V_{2}OV_{3} = ... = V_{n}OV_{1} = \frac{360^{\circ}}{n}$.
Hence, $0 \leq \theta \leq \frac{360^{\circ}}{n}$.
Furthermore, we can limit this range to $0 \leq \theta \leq \frac{180^{\circ}}{n}$, as when $\theta \geq \frac{180^{\circ}}{n}$, the diagram can be flipped to reduce $\theta$.
Now, we can try to find $\angle V_{m+1}OB$. To do this, we find the slice of the polygon which contains segment $OB$. We can find the number of triangles that has to be passed through to form $\angle V_{1}OB$. Let this value be $x$. This gives the expression:

\begin{align*}
	 x = frac{\theta + 90^{\circ}}{frac{360^{\circ}}{n}}
\end{align*}

Simplifying it,

\begin{align*}
	 x & = frac{\theta + 90^{\circ}}{frac{360^{\circ}}{n}}   \\
	 & = frac{\theta n + 90^{\circ} n}{360^{\circ}} \\
	 & = frac{\theta n}{360^{\circ}} + frac{n}{4}
\end{align*}

Plugging in the range for $\theta$,

\begin{align*}
	frac{n}{4} \leq frac{\theta n}{360^{\circ}} + frac{n}{4} \leq frac{n}{4} + frac{1}{2}
\end{align*}

Substituting $n$ for $4m$, 

\begin{align*}
	m \leq x \leq m + frac{1}{2}
\end{align*}

It can be seen that point $B$ is in triangle $V_{m+1}OV_{m+2}$, and is closer to $V_{m+1}$ than $V_{m+2}$.
We can now check for equality between $OA$ and $OB$, as this would render $ABCD$ as a square. To do that, the equality $\angle V_{m+1}OB = \angle V_{1}OA$ must be true.

\begin{align*}
	\angle V_{m+1}OB & = \theta + 90^{\circ} - m\left(\frac{360^{\circ}}{n}\right)   \\
	& = \theta
\end{align*}

Hence, a square can always be inscribed.

To find the maximum side length of the square, we can maximise $OA$. Let $OV_{1} = r$.
Using the sine law, 

\begin{align*}
	\angle AV_{1}O & = 90^{\circ} - \frac{180^{\circ}}{n}     \\
	\frac{OA}{\sin \left(90^{\circ} - \frac{180^{\circ}}{n}\right)} & = \frac{r}{\sin \left(180^{\circ} - \theta - \left(90^{\circ} - \frac{180^{\circ}}{n}\right)\right)}  \\
	\frac{OA}{\cos \frac{180^{\circ}}{n}} & = \frac{r}{\sin \left(\theta + 90^{\circ} - \frac{180^{\circ}}{n}\right)}    \\
	OA & = \frac{r \cos \frac{180^{\circ}}{n}}{\cos \left(\frac{180^{\circ}}{n} - \theta\right)}
\end{align*}

Notice how the numerator is fixed. To maximise $OA$, we need to minimise $\cos \left(\frac{180^{\circ}}{n} - \theta\right)$. We need to maximise $\frac{180^{\circ}}{n} - \theta$, hence we have to minimise $\theta$. This can be done when $\theta = 0$.

Now, we can find the side length of the square, $s$, from $r$, and by expressing $r$ from the side length of the entire polygon, $k$, we can find $s$ from $k$ and $n$.

\begin{align*}
	OA & = \frac{r \cos \frac{180^{\circ}}{n}}{\cos \left(\frac{180^{\circ}}{n}\right)}  \\
	& = r  \\
	s & = OA\sqrt{2} \\
	& = r\sqrt{2} \\
	2r^2 \left(1 - \cos \frac{360^{\circ}}{n}\right) & = k^2  \\
	r & = k\sqrt{\frac{1}{2 - 2\cos\frac{360^{\circ}}{n}}}  \\
	s & = k\sqrt{\frac{2}{2 - 2\cos\frac{360^{\circ}}{n}}}  \\
	& = k\sqrt{\frac{1}{1 - 1\cos\frac{360^{\circ}}{n}}}
\end{align*}

\printbibliography
\end{document}
